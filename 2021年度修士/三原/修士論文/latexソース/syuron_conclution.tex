\chapter{まとめ}
 本研究ではデータストリームに対する類似検索の高速化を念頭に,データストリームに対するハッシュ値の更新アルゴリズムを取り扱った.とくにスライディングウィンドウを動的に変化する多重集合と見なして,スライディングウィンドウに対するMin-Hashのハッシュ値更新アルゴリズムSWMHを提案した.SWMHはスライディングウィンドウで集合を取り扱ったDatarらの手法を多重集合に拡張したものであり,スライディングウインドウモデルで多重集合を取り扱える初の手法である.
  SWMHの特筆すべき点は,動的多重集合に対してMin-hashを計算する場合,1つの要素への割り当て値が多重度に影響されて動的に変化するという難点に対応したことである.さらに要素への割り当て値をハッシュ値が変化しない範囲で修正することにより,SWMHが管理する必要がある要素数を削減させて,実行時間を短縮した.
さらに,SWMHを拡張し,一度に複数個の要素がスライディングウインドウに到着するモデルに対応したMin-hash計算手法であるバッチSWMHを提案した.バッチSWMHでは,要素1つ1つに対してMinlistを更新するのではなく,ウインドウに入ってきた要素群の中で,最小の割り当て値を保持する要素と最大多重度のラベルを保持する要素の2つに絞り,Minlistへアクセスする回数を減らすことで実行時間の短縮を狙った.
 人工datasetと実データセットを用いてMin-hash計算時間を測る実験を行い,提案手法を評価した.まず,SWMHは,毎時刻Min-hashのハッシュ値を再計算するベースラインより,高速にハッシュ値を算出できることを示せた.さらに,バッチSWMHは,複数個の要素がウインドウを出入りする場合にSWMHより短い時間でMin-hashを高速計算できることを示した.
 最後に,提案手法SWMHでは,データを保持するためにヒストグラムを多用しているため,多くのメモリを消費している.従って,今後の研究課題としては,メモリ使用量の削減のために近似ヒストグラムを用いてハッシュ値を計算する手法の実現が望まれる.