本研究では,久保ら\cite{kubo}が提唱したストリーム環境で動的に変化するテキスト
集合を対象とした類似検索であるCTS問題を取り扱った.そして,CTS問題
を解く既存手法(遅延評価法)の転置インデクスを用いた高速化に取り組んだ.

CTS問題では,クエリとデータベースの両者が時間と共に変化するため,更新
オーバーヘッドを考慮して転置インデクスを導入することが求められる.
本稿では,クエリテキスト集合$V_T$にのみ転置インデクスを構築し,データベース側
のテキスト集合$U_T$には転置インデクスを保持しないLE-Q
%(Lazy Evaluation with Query-side inverted indices)
の方が,クエリとデータベース両者に転置インデクスを用意する方式 LE-QD よりも処理時間
が短縮でき,適切であることを示した.これはデータベース内の多数のテキストストリ
ームに対して転置インデクスを作成すると,更新オーバーヘッドが膨大になるためである.
特に,データベース側のテキスト集合に転置インデクスを用意すると,転置インデクスを
使用しない場合よりも処理速度が遅くなる場合があることも実験的に確認した.

一方でLE-Qでは,クエリテキスト集合$V_T$にのみ転置インデクスを準備するため,
$V_T$内のテキスト$V_T$を起点とするマッチング判定では転置インデクスを
使えない.しかし,LE-Qではそのようなマッチング判定も多数実行される.
LE-Qをさらに高速化するため,遅延評価法で枝刈り効率が悪い状況で
$V_T$を起点とするマッチング判定を$U_T$を起点とするマッチング判定
に適応的に切り替える手法 ALE-Qを提案した.$U_T$を起点としてマッチング
判定した場合,類似度を厳密計算することになり枝刈り効率は低下する一方で,
マッチング判定そのものは転置インデクスにより高速化される.人工データ,実データを用いた
実験により,ALE-Qが遅延評価法,LE-Qよりも実行速度が早くなることを示した.

今後の課題としては,ALE-Qでマッチング判定方法を切り替える条件の最適化が挙げられる.
