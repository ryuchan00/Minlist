% !TEX root = Sotsuron_main_v3.tex
%%Chap1
\section{まえがき}
%現在,高度情報化する社会においてSNSなどのテキストストリームに対する類似検索は重要性が高まってきている.ユーザ間の類似度を知ることで,ユーザエクスペリエンスを向上させる適切なコンテンツの提供や,不適切なコンテンツをもつユーザの発見が可能になると考える.SNSには多くのユーザが存在しており,多くのユーザ間の類似性を判断するには莫大な時間を必要とする.そこで本研究では従来手法に対して転置インデクスを用いることで高速な類似検索を実現させた.

% 近年,ソーシャルネットワークやIoT (Internet of Things)の発展に伴い,ス
% トリームデータ解析の重要性が高まっている.その中でストリームデータを対
% 象とする類似検索は,リアルタイムでの情報推薦や異常検出の基盤技術として
% 注目されている.これまでのストリームデータを対象とした類似検索は,データストリームをデータベースをみなし,データベース内のデータが増減する環境を多く取り扱っている\cite{Leong}\cite{Kyriakos}\cite{Di}.とくに最近は,データストリームをデータの集合と捉え,
% 集合間類似検索により類似データストリームを検索する問題設定が取り扱われて
% いる.例えば,Xuら\cite{xu}は1データストリームのスライディングウィンドウ
% 内の要素群をクエリ集合として,(時間によらない静的な)集合のデータベース
% からクエリと最も類似した上位$k$個の集合を探す検索問題を提唱した.逆に
% Kogaら\cite{noguchi}では,データベースに複数のデータストリームが登録された状況で,
% (時間によらない静的な) 集合をクエリとして,スライディングウィンドウの内容
% が最も類似した上位$k$個のデータストリームを検索する問題を取り扱った.
% これらの問題では,時間経過によりデータストリームに新しい要素が来ると,
% スライディングウィンドウの内容が変わるため,検索結果を随時更新する必要が
% ある(Continuous Similarity Search). 

Efstathiadesら\cite{efstathiades},Kuboら\cite{kubo}はSNSから類似ユーザを探すことを想定し,
ユーザ$U$を投稿したテキスト群でモデル化し,テキスト集合間の類似検索を取り扱った.その中にCTS問題\cite{kubo}がある.CTS問題とは,ユーザ$U$のテキストストリーム$X_U$のスライディングウィンドウをクエリテキスト集合とし,
スライディングウィンドウの類似度が閾値$\epsilon_u$以上となるテキストストリームを
データベースから探索するレンジ探索問題である.
ここで$X_U$はユーザ$U$が投稿したテキストの集合であり,時間と共に新しいテキストが追加される.
$X_U$のスライディングウィンドウは$U$の最近の投稿内容を表す.そして,スライディングウィンドウが似たテキストストリームを探すことにより,$U$と最近の投稿内容が似た他のユーザを見つけられる.
久保ら\cite{kubo}はCTS問題に対して遅延評価法という枝刈りベースの高速アルゴリズムを提案した.
しかし,遅延評価法ではテキスト検索でよく使われる転置インデクスを使用しておらず,多くの処理時間が必要である.


%本稿ではCTS問題を研究対象とし,遅延評価法を転置インデクスにより高速化することを目指す.
本研究ではCTS問題を研究対象とし,遅延評価法を転置インデクスにより高速化することを目指す.
転置インデクスはテキスト検索における標準的な要素技術である.しかし,CTS問題ではテキスト集合
が動的に変化するため,転置インデクスの更新オーバーヘッドを考慮する必要があり,転置インデクスの
適用は自明ではない.本研究では,まず遅延評価法で起こるテキストマッチングのパターンを分析し,
次に分析結果を踏まえマッチング時に転置インデクスを使用するか否かを切り替える手法を提案
する.そして,切り替えを適切に行うことで転置インデクスを用いて遅延評価法を転置インデクスを使わ
なかった場合より高速化できることを実験評価により示す.なお,切り替え方式が適切でないと,転置
インデクス更新のオーバーヘッドのため,実行速度が却って遅くなる場合があることも確認した.
本研究の新規性はクエリとデータベースの両側でテキスト集合が動的に変化する条件で,転置インデクス
の適切な使用法を示したことである.


\section{本論文の構成}
以下に本稿の構成を述べる.2章でCTS 問題の定義を延べ,3章でベースラインとなる
遅延評価法を記述する.4章では転置インデクスについて簡潔に説明し,5章では
提案手法となる遅延評価法の転置インデクスによる高速化方式を記述する.
6章では提案手法に改善を加えた手法を示し,7章で提案手法を人工データと実データを用いて実験的に評価し,考察する.8章は結論である.

5章の提案手法のLE-Q,LE-QDは第174回データベースシステム研究発表会で発表済み\cite{me}であるが,本稿では提案手法として記述し,提案手法の改善案であるALE-Qに関しては内容を変更し,再定義する.