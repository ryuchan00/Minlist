% !TEX root = main.tex
% おまじない:上の一文を足すことで,分割したどのtexからもコンパイル可能となる
%% Sotsuron_preamble
\usepackage{indentfirst}
\usepackage[dvipdfmx]{graphicx}
\usepackage{afterpage} % 図が章をまたがない
\usepackage{here}
\usepackage{url}
\usepackage{seqsplit} % 長い単語の改行
\usepackage{comment}

\usepackage[dvipdfmx]{hyperref} %目次リンク
\usepackage{pxjahyper} %目次リンク

\usepackage[dvipdfmx]{color}
%\usepackage{listings, jlisting}

\usepackage{amsmath}
\usepackage{algorithm, algpseudocode}
\usepackage{listings}

\usepackage{enumerate}

\usepackage{otf}


\usepackage{theorem}



\usepackage{url}
\newtheorem{theo}{定理}
\newtheorem{defi}{定義}
\newtheorem{lemm}{補題}
\newtheorem{proof}{証明}
\lstset{
	language=C++,
	classoffset=0,% 環境0
	basicstyle=\ttfamily\scriptsize,
	%commentstyle=\textit,
	commentstyle = {\itshape \color[cmyk]{1,0.4,1,0}},
	%keywordstyle=\bfseries,
	keywordstyle = {\bfseries \color{magenta}},%int,ifなどのスタイル
	stringstyle = {\ttfamily \color{blue}},% string ""で囲まれたやつのスタイル
	frame=tRBl,
	framesep=5pt,%frameまでの間隔(行番号とプログラムの間)
	showstringspaces=false,
	numbers=left,%行番号の位置
	stepnumber=1,%行番号の間隔
	numberstyle=\tiny,%行番号の書体
	tabsize=2,
	breaklines=true,%行が長くなった場合の改行
	morecomment=[l][\color{magenta}]{\#},%comment追加
	% starting new class
	classoffset=1,% 環境1
	otherkeywords={.,!,=,+,-,>,<},
	morekeywords={.,!,=,+,-,>,<},
	keywordstyle=\color{red},
	sensitive  = true,
	%classoffset=2,
	%otherkeywords={->},
	%alsodigit={-},
	%morekeywords={->},
	%keywordstyle=\color{magenta},
	classoffset=2,
	morekeywords={%keywordの追加
		uint32_t, uint64_t, uint16_t%
	},
	keywordstyle=\color{blue}
}

\makeatletter
\newenvironment{breakablealgorithm}
  {% \begin{breakablealgorithm}
   \begin{center}
     \refstepcounter{algorithm}% New algorithm
     \hrule height.8pt depth0pt \kern2pt% \@fs@pre for \@fs@ruled
     \renewcommand{\caption}[2][\relax]{% Make a new \caption
       {\raggedright\textbf{\ALG@name~\thealgorithm} ##2\par}%
       \ifx\relax##1\relax % #1 is \relax
         \addcontentsline{loa}{algorithm}{\protect\numberline{\thealgorithm}##2}%
       \else % #1 is not \relax
         \addcontentsline{loa}{algorithm}{\protect\numberline{\thealgorithm}##1}%
       \fi
       \kern2pt\hrule\kern2pt
     }
  }{% \end{breakablealgorithm}
     \kern2pt\hrule\relax% \@fs@post for \@fs@ruled
   \end{center}
  }
\makeatother


% remove end sentences
\algtext*{EndWhile}
\algtext*{EndIf}
\algtext*{EndFor}
\algtext*{EndLoop}

% renew symbol
\algrenewcommand\algorithmicrequire{\textbf{Input:}}
\algrenewcommand\algorithmicensure{\textbf{Outout:}}
\newcommand{\abs}[1]{\lvert #1 \rvert}
\newcommand{\Break}{\textbf{break}}
%\usepackage{geometry}

\usepackage{wuse_thesis}

\bibliographystyle{junsrt}
%\bibliographystyle{junsrtquotationmarks} % junsrt 引用順で表示

\title{動的なテキスト集合に対する類似検索\\アルゴリズムALE-Qの評価}  %タイトル

\author{土田 祐将}  %名前
\bachelar
\department{\ajRoman{1}}   %学科
\course{コンピュータサイエンス}   %プログラム
\studentid{1810430}    %学籍番号
\teacher{古賀 久志 准教授}  %指導教員
\gyear{令和3}
\date{令和4年1月31日}